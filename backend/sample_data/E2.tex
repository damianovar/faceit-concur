\begin{frame}{Question E2.1}
	\QuestionKCs{biofeedback}
	\QuestionKCsTaxonomies{(1,1)}
	\QuestionNotes{}
	\QuestionBody{What is a biofeedback device?}
	\QuestionAnswers
	{
		\answer any medical device for measuring some physiological functions
		\correctanswer any medical device for gaining awareness of some physiological functions
		\answer any medical device for visualizing some physiological functions
		\answer I do not know
	}
\end{frame}


\begin{frame}{Question E2.2}
	\QuestionKCs{population validity}
	\QuestionKCsTaxonomies{(1,1)}
	\QuestionNotes{}
	\QuestionBody{Using data from Scalable-Learning, you model your retention of various concepts within probability theory. You then use this model to optimize the order in which you should have studied the Knowledge Components (KCs) to minimize your learning efforts. You give this curriculum to another friend who took the course, but who comes from a completely different background. She looks at your curriculum and says ``this learning path makes no sense''. What is missing here?}
	\QuestionAnswers
	{
		\answer internal validity
		\answer external validity
		\correctanswer population validity
		\answer ecological validity
		\answer I do not know
	}
\end{frame}


\begin{frame}{Question E2.3}
	\QuestionKCs{internal validity,external validity}
	\QuestionKCsTaxonomies{(1,1),(1,1)}
	\QuestionNotes{}
	\QuestionBody{Internal validity deals with the ``outputs'' of the system, external validity deals with the ``inputs'' of the system.}
	\QuestionAnswers
	{
		\answer true
		\correctanswer false
		\answer I do not know
	}
\end{frame}


\begin{frame}{Question E2.4}
	\QuestionKCs{ecological validity}
	\QuestionKCsTaxonomies{(1,1)}
	\QuestionNotes{}
	\QuestionBody{A model is ecologically valid when}
	\QuestionAnswers
	{
		\answer it helps solve environmental problems
		\correctanswer it can be used to describe what happens in real life settings
		\answer it can be used to describe the effects of real life settings
		\answer I do not know
	}
\end{frame}


\begin{frame}{Question E2.5}
	\QuestionKCs{population validity}
	\QuestionKCsTaxonomies{(1,1)}
	\QuestionNotes{here put your notes, if you want}
	\QuestionBody{Using data from a single oil platform, I create a model to predict the daily production. Then I try to use that model on another platform. What may I risk?}
	\QuestionAnswers
	{
		\answer to do not have internal validity
		\answer to do not have external validity
		\correctanswer to do not have population validity
		\answer to do not have ecological validity
		\answer I do not know
	}
\end{frame}


\begin{frame}{Question E2.6}
	\QuestionKCs{KLI framework}
	\QuestionKCsTaxonomies{(1,1)}
	\QuestionNotes{}
	\QuestionBody{Why are we introducing the Knowledge Learning Instruction (KLI) framework?}
	\QuestionAnswers
	{
		\answer to teach a bit about pedagogy
		\answer to be able to provide feedback in a quantitative fashion
		\correctanswer to manage feedback using a quantitative approach
		\answer to be able to present feedback in the ``reference-groups''
		\answer I do not know
	}
\end{frame}


\begin{frame}{Question E2.7}
	\QuestionKCs{attention check}
	\QuestionKCsTaxonomies{(1,1)}
	\QuestionNotes{attention check}
	\QuestionBody{The previous question was about internal vs.\ external validity}
	\QuestionAnswers
	{
		\answer true
		\correctanswer false
		\answer I do not know
	}
\end{frame}


\begin{frame}{Question E2.8}
	\QuestionKCs{taxonomy}
	\QuestionKCsTaxonomies{(1,1)}
	\QuestionNotes{this only to see if a person is paying attention to what is being said in class that is not written in the slides}
	\QuestionBody{What is the defining characteristic of a ``Knowledge Component'' within the KLI framework?}
	\QuestionAnswers
	{
		\correctanswer being assessable
		\answer being an atomic knowledge unit
		\answer being part of an intended learning outcome
		\answer being explainable in a quantitative fashion
		\answer I do not know
	}
\end{frame}


\begin{frame}{Question E2.9}
	\QuestionKCs{taxonomy}
	\QuestionKCsTaxonomies{(1,1)}
	\QuestionNotes{}
	\QuestionBody{What does the word ``taxonomy'' mean?}
	\QuestionAnswers
	{
		\answer a glossary of things or concepts
		\correctanswer a classification of things or concepts
		\answer a set of terminologies for things or concepts
		\answer I do not know
	}
\end{frame}


\begin{frame}{Question E2.10}
	\QuestionKCs{taxonomy}
	\QuestionKCsTaxonomies{(1,1)}
	\QuestionNotes{}
	\QuestionBody{What do ``taxonomies'' enable within the KLI framework?}
	\QuestionAnswers
	{
		\correctanswer to sort different levels of understanding
		\answer to create graphical representations of understanding
		\answer to understand the connections between different types of understanding
		\answer I do not know
	}
\end{frame}


\begin{frame}{Question E2.11}
	\QuestionKCs{prior distribution}
	\QuestionKCsTaxonomies{(1,1)}
	\QuestionNotes{}
	\QuestionBody{A prior probability distribution is}
	\QuestionAnswers
	{
		\answer a data-dependent distribution that captures the domain knowledge
		\answer a conditional probability assigned before some evidence is taken into account
		\correctanswer a belief about a random variable before some evidence is taken into account
		\answer I do not know
	}
\end{frame}


\begin{frame}{Question E2.12}
	\QuestionKCs{conjugate prior distribution}
	\QuestionKCsTaxonomies{(1,1)}
	\QuestionNotes{}
	\QuestionBody{A conjugate prior probability distribution is a prior probability distribution that:}
	\QuestionAnswers
	{
		\correctanswer when combined with a specific likelihood function it produces a posterior that is in the same family of the original prior
		\answer when combined with a normal likelihood function produces a posterior that is in the same family of the original prior
		\answer when combined with a normal likelihood function produces a posterior that is equal to the original prior
		\answer I do not know
	}
\end{frame}


\begin{frame}{Question E2.13}
	\QuestionKCs{joint distribution}
	\QuestionKCsTaxonomies{(1,1)}
	\QuestionNotes{}
	\QuestionBody{What is a joint probability distribution?}
	\QuestionAnswers
	{
		\answer a distribution that gives probabilities over some random variables' values
		\correctanswer a distribution that gives probabilities over combinations of multiple random variables' values
		\answer a distribution that gives the likelihood of multiple random variables' values
		\answer I do not know
	}
\end{frame}


\begin{frame}{Question E2.14}
	\QuestionKCs{marginal distribution}
	\QuestionKCsTaxonomies{(1,1)}
	\QuestionNotes{}
	\QuestionBody{What is a marginal probability distribution?}
	\QuestionAnswers
	{
		\answer the probability that one obtains by eliminating some variables in a joint probability distribution
		\answer the probability that one obtains by emarginating some variables in a joint probability distribution
		\correctanswer the probability that one obtains by integrating some variables in a joint probability distribution
		\answer I do not know
	}
\end{frame}


\begin{frame}{Question E2.15}
	\QuestionKCs{marginal distribution,joint distribution,independence}
	\QuestionKCsTaxonomies{(1,1)}
	\QuestionNotes{}
	\QuestionBody{If one knows all the associated marginal probability distributions $p(x_{i})$ then one can also build the associated joint distribution $p(x_{1}, \ldots, x_{n})$.}
	\QuestionAnswers
	{
		\answer true
		\answer false
		\answer only for Gaussian random variables
		\correctanswer only for independent random variables
		\answer I do not know
	}
\end{frame}


\begin{frame}{Question E2.16}
	\QuestionKCs{independence,joint distribution}
	\QuestionKCsTaxonomies{(1,1)}
	\QuestionNotes{}
	\QuestionBody{The fact that two random variables are independent means that\ldots}
	\QuestionAnswers
	{
		\answer their marginals are orthogonal
		\correctanswer their joint distribution is the product of their marginals
		\answer their joint distribution contains orthogonal marginals
		\answer I do not know
	}
\end{frame}


\begin{frame}{Question E2.17}
	\QuestionKCs{conditional probability}
	\QuestionKCsTaxonomies{(1,1)}
	\QuestionNotes{}
	\QuestionBody{$P(A|B) = \ldots$}
	\QuestionAnswers
	{
		\answer $\frac{P(A \cup B)}{P(B)}$
		\correctanswer $\frac{P(A \cap B)}{P(B)}$
		\answer $\frac{P(A \perp B)}{P(B)}$
		\answer I do not know
	}
\end{frame}


\begin{frame}{Question E2.18}
	\QuestionKCs{internal validity,external validity}
	\QuestionKCsTaxonomies{(1,1),(1,1)}
	\QuestionNotes{}
	\QuestionBody{Mechanistic models / physics based models are}
	\QuestionAnswers
	{
		\answer Generalizable
		\answer Trustworthy
		\answer Accounting only for partial physics
		\answer Potentially biased
		\correctanswer All the qualities above
		\answer I do not know
	}
\end{frame}


\begin{frame}{Question E2.19}
	\QuestionKCs{internal validity,external validity}
	\QuestionKCsTaxonomies{(1,1),(1,1)}
	\QuestionNotes{}
	\QuestionBody{Data-driven models}
	\QuestionAnswers
	{
		\answer Can adapt in time
		\answer Are trustworthy, because data implicitly captures physics
		\answer Can be biased
		\answer only 1 and 2 above
		\correctanswer only 1 and 3 above
		\answer I do not know
	}
\end{frame}

\begin{frame}{Question E2.20}
	\QuestionKCs{internal validity,external validity}
	\QuestionKCsTaxonomies{(1,1),(1,1)}
	\QuestionNotes{}
	\QuestionBody{Interpretability of data-driven models generally \ldots\ldots with complexity and accuracy \ldots\ldots. Fill the blanks}
	\QuestionAnswers
	{
		\correctanswer decrease, increase
		\answer increase, decrease
		\answer decrease, decrease
		\answer increase, increase
		\answer I do not know
	}
\end{frame}


\begin{frame}{Question E2.21}
	\QuestionKCs{internal validity,external validity}
	\QuestionKCsTaxonomies{(1,1),(1,1)}
	\QuestionNotes{}
	\QuestionBody{Categorical variables can be dealt with}
	\QuestionAnswers
	{
		\answer just converting them into numbers
		\correctanswer converting them as one-hot-encoded vectors
		\answer both 1 and 2
		\answer nothing. It is not possible to work with categorical variables in a data-driven approach
		\answer I do not know
	}
\end{frame}


\begin{frame}{Question E2.22}
	\QuestionKCs{internal validity,external validity}
	\QuestionKCsTaxonomies{(1,1),(1,1)}
	\QuestionNotes{}
	\QuestionBody{I have a simple linear model $y=Ax_1+Bx_2+D$. I can be sure that if $A$ is positive, then if $x_1$ increases \ldots}
	\QuestionAnswers
	{
		\answer $y$ may increase
		\answer $y$ may decrease
		\answer $y$ must increase
		\correctanswer 1 and 2
		\answer I do not know
	}
\end{frame}

\begin{frame}{Question E2.23}
	\QuestionKCs{attention check}
	\QuestionKCsTaxonomies{(1,1)}
	\QuestionNotes{attention check}
	\QuestionBody{The previous question was about internal vs.\ external validity}
	\QuestionAnswers
	{
		\answer true
		\correctanswer false
		\answer I do not know
	}
\end{frame}

\begin{frame}{Question E2.24}
	\QuestionKCs{likelihood}
	\QuestionKCsTaxonomies{(1,1)}
	\QuestionNotes{}
	\QuestionBody{The likelihood function is \ldots}
	\QuestionAnswers
	{
		\answer a function of the data
		\answer a function of the parameters
		\correctanswer a function of both
		\answer I do not know
	}
\end{frame}


\begin{frame}{Question E2.25}
	\QuestionKCs{covariance}
	\QuestionKCsTaxonomies{(1,1)}
	\QuestionNotes{}
	\QuestionBody{Assume to have a Gaussian r.v.\ whose covariance matrix determinant is big. Then the associated probability density \ldots}
	\QuestionAnswers
	{
		\correctanswer has a wide peak
		\answer has a tall peak
		\answer has a skewed peak
		\answer I do not know
	}
\end{frame}


\begin{frame}{Question E2.26}
	\QuestionKCs{centrality}
	\QuestionKCsTaxonomies{(1,1)}
	\QuestionNotes{}
	\QuestionBody{Among the 3 most typical ``centrality measures'' (i.e., mean, median, and mode) which measures are more or less sensitive to the tails of the distributions for which they are computed?}
	\QuestionAnswers
	{
		\correctanswer mean = most sensitive, mode = least sensitive
		\answer median = most sensitive, mode = least sensitive
		\answer mode = most sensitive, mean = least sensitive
		\answer median = most sensitive, mean = least sensitive
		\answer I do not know
	}
\end{frame}


\begin{frame}{Question E2.27}
	\QuestionKCs{dispersion}
	\QuestionKCsTaxonomies{(1,1)}
	\QuestionNotes{}
	\QuestionBody{Among the 2 most typical ``dispersion measures'' (i.e., standard deviation and interquartile ranges) which are more or less sensitive to the tails of the distributions for which they are computed?}
	\QuestionAnswers
	{
		\correctanswer standard deviation = most sensitive, interq.\ range = least sensitive
		\answer standard deviation = least sensitive, interq.\ range = most sensitive
		\answer equally sensitive
		\answer I do not know
	}
\end{frame}


\begin{frame}{Question E2.28}
	\QuestionKCs{centrality}
	\QuestionKCsTaxonomies{(1,1)}
	\QuestionNotes{}
	\QuestionBody{Which class is over-represented in the two phase flow dataset?}
	\QuestionAnswers
	{
		\answer B=bubble
		\correctanswer I=intermittent
		\answer A=annular
		\answer DB=disperse bubble
		\answer SS=stratified smooth
		\answer SW=stratified wavy.
		\answer I do not know
	}
\end{frame}


\begin{frame}{Question E2.29}
	\QuestionKCs{vaginal pressure dataset}
	\QuestionKCsTaxonomies{(1,1)}
	\QuestionNotes{}
	\QuestionBody{How many pressure sensors are present in the ``vaginal pressure dataset''?}
	\QuestionAnswers
	{
		\answer 6
		\correctanswer 8
		\answer 10
		\answer I do not know
	}
\end{frame}


\begin{frame}{Question E2.30}
	\QuestionKCs{hyperspectral images dataset}
	\QuestionKCsTaxonomies{(1,1)}
	\QuestionNotes{}
	\QuestionBody{How many wavelengths are measured in the ``hyperspectral images dataset''?}
	\QuestionAnswers
	{
		\correctanswer 220
		\answer 256
		\answer 512
		\answer 550
		\answer I do not know
	}
\end{frame}


\begin{frame}{Question E2.31}
	\QuestionKCs{happiness index dataset}
	\QuestionKCsTaxonomies{(1,1)}
	\QuestionNotes{}
	\QuestionBody{Which information is \ItPrimary{not} present in the ``happiness index dataset''?}
	\QuestionAnswers
	{
		\answer freedom
		\answer social support
		\answer GDP per capita
		\correctanswer stress
		\answer I do not know
	}
\end{frame}

