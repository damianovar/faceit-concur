\begin{frame}{Question E1.1}
	\QuestionKCs{complex numbers}
	\QuestionKCsTaxonomies{(1,1)}
	\QuestionBody{The phase angle of $3-j4$, where $j$ is the imaginary unit, is}
	\QuestionAnswers
	{
		\answer $5$
		\answer $-4$
		\correctanswer $\mathrm{arctan} \left(\frac{-4}{3}\right)$
		\answer $\cos(3)-j\sin(4)$
		\answer none of the above
		\answer I do not know
	}
	\QuestionSolution{in the complex plane, the number is identified by (3,-4), and the arctan solution follows immediately}
	\QuestionAuthor{}
	\QuestionVersion{}
\end{frame}


\begin{frame}{Question E1.2}
	\QuestionKCs{powers,logarithms}
	\QuestionKCsTaxonomies{(1,1)}
	\QuestionBody{${x^3}^2$ is equal to}
	\QuestionAnswers
	{
		\answer $x^1$
		\answer $x^5$
		\correctanswer $x^6$
		\answer $x^9$
		\answer none of the above
		\answer I do not know
	}
	\QuestionSolution{}
	\QuestionAuthor{}
	\QuestionVersion{}
\end{frame}


\begin{frame}{Question E1.3}
	\QuestionKCs{complex numbers}
	\QuestionKCsTaxonomies{(1,1)}
	\QuestionBody{The magnitude of $e^{j a^2}$, where $j$ is the imaginary unit, is}
	\QuestionAnswers
	{
		\answer $0$
		\correctanswer $1$
		\answer $a$
		\answer $a^2$
		\answer none of the above
		\answer I do not know
	}
	\QuestionSolution{the number is on the unitary circle, thus the magnitude is 1}
	\QuestionAuthor{}
	\QuestionVersion{}
\end{frame}


\begin{frame}{Question E1.4}
	\QuestionKCs{complex numbers}
	\QuestionKCsTaxonomies{(2,1)}
	\QuestionBody
	{
		The roots of a generic polynomial equation
		$a_{0} + a_{1} x + \ldots + a_{n} x^{n} = 0$
		with $a_{0}, \ldots, a_{n} \in \Reals$ are 
	}
	\QuestionAnswers
	{
		\answer $n$, distinct, and in $\Reals$
		\answer $n$, distinct, and in $\Complex$
		\answer at most $n$, distinct, and always in $\Reals$
		\correctanswer at most $n$, distinct, and potentially in $\Complex$
		\answer none of the above
		\answer I do not know
	}
	\QuestionSolution{this is the fundamental theorem of algebra}
	\QuestionAuthor{}
	\QuestionVersion{}
\end{frame}


\begin{frame}{Question E1.5}
	\QuestionKCs{roots of polynomials}
	\QuestionKCsTaxonomies{(1,1)}
	\QuestionBody{There exist higher-degree real-valued polynomial equations that have no solutions.}
	\QuestionAnswers
	{
		\answer true
		\correctanswer false
		\answer I do not know
	}
	\QuestionSolution{from the fundamental theorem of algebra we know that there must exist at least one solution (real or pair of complex conjugates)}
	\QuestionAuthor{}
	\QuestionVersion{}
\end{frame}


\begin{frame}{Question E1.6}
	\QuestionKCs{systems of linear equations}
	\QuestionKCsTaxonomies{(1,1)}
	\QuestionBody{Given the set of linear equations $A x = b$ with $A \in \Reals^{n \times m}$, $x\in\Reals^{m\times 1}$ and $b\in\Reals^{n\times 1}$, when does a unique solution $x$ exist?}
	\QuestionAnswers
	{
		\answer always, if $n=m$
		\correctanswer potentially, if $n\geq m$ and depending on $A$
		\answer none of the above
		\answer I do not know
	}
	\QuestionSolution{}
	\QuestionAuthor{}
	\QuestionVersion{}
\end{frame}


\begin{frame}{Question E1.7}
	\QuestionKCs{systems of linear equations}
	\QuestionKCsTaxonomies{(1,1)}
	\QuestionBody
	{
		Consider the system
		$$
		\left\lbrace
		\begin{array}{rrrl}
			x & + 3 y & + z & = 3 \\
			2 x & + y & + 2 z & = 3 \\
			x & + 3 y & + 2 z & = 3 \\
		\end{array}
		\right.
		$$
		To obtain an equivalent system (i.e., a system that admits the same solutions of the original one) one may take the first row and substitute it with the same row \ldots
	}
	\QuestionAnswers
	{
		\answer multiplied by a generic number in $\Reals$
		\answer summed with an other row
		\correctanswer all the above operations
		\answer none of the above operations
		\answer I do not know
	}
	\QuestionSolution{}
	\QuestionAuthor{}
	\QuestionVersion{}
\end{frame}


\begin{frame}{Question E1.8}
	\QuestionKCs{matrix multiplication}
	\QuestionKCsTaxonomies{(1,1)}
	\QuestionBody
	{
		With which of the following matrices can
		$A=\begin{bmatrix} 1 & -7 & 1\\ 1 & 0 & 5\end{bmatrix}$
		be right-multiplied with?
	}
	\QuestionAnswers
	{
		\answer $\begin{bmatrix} 1 & 0 \\ 1 & -1 \end{bmatrix}$
		\answer $\begin{bmatrix} 1 & 0 & -3 \\ 1 & -1 & 0 \end{bmatrix}$
		\answer $\begin{bmatrix} 1 & 0 \end{bmatrix}$
		\correctanswer $\begin{bmatrix} 1 & 8 & -3 \\ 1 & -1 & 0 \\ 0 & 1 & 1\end{bmatrix}$
	}
	\QuestionSolution{}
	\QuestionAuthor{}
	\QuestionVersion{}
\end{frame}


\begin{frame}{Question E1.9}
	\QuestionKCs{determinants}
	\QuestionKCsTaxonomies{(1,1)}
	\QuestionBody
	{
		The determinant of 
		$A=\begin{bmatrix} A' & 0 \\ 1 & 0\end{bmatrix}$
		is equal to
	}
	\QuestionAnswers
	{
		\correctanswer $0$
		\answer $1$
		\answer $\det(A')$
		\answer $\det(A')+1$
		\answer none of the above
		\answer I do not know
	}
	\QuestionSolution{}
	\QuestionAuthor{}
	\QuestionVersion{}
\end{frame}


\begin{frame}{Question E1.10}
	\QuestionKCs{injectivity,surjectivity,linearity}
	\QuestionKCsTaxonomies{(1,1),(1,1),(1,1)}
	\QuestionBody
	{
		Let $A \in \Reals^{n \times n}$ be s.t.\ $\det \left( A \right) = 0$. Consider the map $y = A x$ so that $y_{1} = A x_{1}$ and $y_{2} = A x_{2}$. Then this map is
	}
	\QuestionAnswers
	{
		\answer injective (i.e., $y_1 = y_2 \implies x_1 = x_2$)
		\answer surjective (i.e., for every $y \in \Reals^n$ exists $x \in \Reals^{n}$ s.t.\ $y = Ax$)
		\answer bijective (i.e., injective + surjective)
		\correctanswer none of the above
		\answer I do not know
	}
	\QuestionSolution{}
	\QuestionAuthor{}
	\QuestionVersion{}
\end{frame}


\begin{frame}{Question E1.11}
	\QuestionKCs{injectivity,surjectivity,linearity}
	\QuestionKCsTaxonomies{(1,1),(1,1),(1,1)}
	\QuestionBody{Let $A \in \Reals^{3 \times 2}$. Consider the map $y = A x$ so that $y_{1} = A x_{1}$ and $y_{2} = A x_{2}$. Then this map is}
	\QuestionAnswers
	{
		\answer injective (i.e., $y_{1} = y_{2} \implies x_{1} = x_{2}$)
		\answer surjective (i.e., for every $y \in \Reals^{n}$ exists $x \in \Reals^{n}$ s.t.\ $y = A x$)
		\answer bijective (i.e., injective + surjective)
		\correctanswer none of the above
		\answer I do not know
	}
	\QuestionSolution{}
	\QuestionAuthor{}
	\QuestionVersion{}
\end{frame}


\begin{frame}{Question E1.12}
	\QuestionKCs{eigenvectors}
	\QuestionKCsTaxonomies{(1,1)}
	\QuestionBody{The set of eigenvectors of a generic matrix $A \in \Reals^{n \times n}$ forms a basis for $\Reals^{n}$.}
	\QuestionAnswers
	{
		\answer yes, always
		\answer no, never
		\correctanswer it depends on $A$
		\answer I do not know
	}
	\QuestionSolution{}
	\QuestionAuthor{}
	\QuestionVersion{}
\end{frame}


\begin{frame}{Question E1.13}
	\QuestionKCs{projection matrices}
	\QuestionKCsTaxonomies{(1,1)}
	\QuestionBody{At least one of the eigenvalues $\lambda_{i}$ of a projection matrix $P \in \Reals^{n \times n}$ is \ldots}
	\QuestionAnswers
	{
		\correctanswer $\lambda_{i} = 0$
		\answer $\lambda_{i} = 1$
		\answer it depends on $P$
		\answer I do not know
	}
	\QuestionSolution{}
	\QuestionAuthor{}
	\QuestionVersion{}
\end{frame}


\begin{frame}{Question E1.14}
	\QuestionKCs{similar matrices}
	\QuestionKCsTaxonomies{(1,1)}
	\QuestionBody
	{
		If two matrices $A$ and $B$ are similar, i.e., if
		$A = P^{-1} B P$
		for some invertible matrix $P$, then they share the same trace
	}
	\QuestionAnswers
	{
		\correctanswer yes, always
		\answer no, never
		\answer it depends on $A$ and $B$
		\answer I do not know
	}
	\QuestionSolution{}
	\QuestionAuthor{}
	\QuestionVersion{}
\end{frame}


\begin{frame}{Question E1.15}
	\QuestionKCs{Jordan forms,eigenspaces}
	\QuestionKCsTaxonomies{(1,1),(1,1)}
	\QuestionBody
	{
		Consider the matrix
		\begin{small}
		$
		A = 
		\begin{bmatrix}
			1 & 1 & 0 & 0 & 0 & 0 & 0 \\
			0 & 1 & 0 & 0 & 0 & 0 & 0 \\
			0 & 0 & 1 & 0 & 0 & 0 & 0 \\
			0 & 0 & 0 & 1 & 1 & 0 & 0 \\
			0 & 0 & 0 & 0 & 1 & 1 & 0 \\
			0 & 0 & 0 & 0 & 0 & 1 & 1 \\
			0 & 0 & 0 & 0 & 0 & 0 & 1 \\
		\end{bmatrix} .
		$
		\end{small}
		The eigenspace formed by the eigenvectors associated to the eigenvalue 1 has dimension \ldots
	}
	\QuestionAnswers
	{
		\answer 1
		\answer 2
		\correctanswer 3
		\answer 4
		\answer none of the above
		\answer I do not know
	}
	\QuestionSolution{}
	\QuestionAuthor{}
	\QuestionVersion{}
\end{frame}


\begin{frame}{Question E1.16}
	\QuestionKCs{stability,linear systems}
	\QuestionKCsTaxonomies{(1,1),(1,1)}
	\QuestionBody
	{
		Let the square matrix $A$ have one eigenvalue $\lambda_{i} = 2$. Then iterating
		$$x(k+1) = A x(k), \qquad k = 0, 1, \ldots$$
	}
	\QuestionAnswers
	{
		\answer $\lim_{k \rightarrow +\infty} \left\| x(k) \right\| = 0$ for every $x(0) \neq 0$
		\answer $\lim_{k \rightarrow +\infty} \left\| x(k) \right\| = 0$ for opportune $x(0) \neq 0$
		\answer $\lim_{k \rightarrow +\infty} \left\| x(k) \right\| = + \infty$ for every $x(0) \neq 0$
		\correctanswer $\lim_{k \rightarrow +\infty} \left\| x(k) \right\| = + \infty$ for opportune $x(0) \neq 0$
		\answer I do not know
	}
	\QuestionSolution{}
	\QuestionAuthor{}
	\QuestionVersion{}
\end{frame}


\begin{frame}{Question E1.17}
	\QuestionKCs{bases}
	\QuestionKCsTaxonomies{(1,1)}
	\QuestionBody{If a set of vectors $v_{1}, \ldots, v_{k}$ in $\Reals^n$ is linearly independent then \ldots}
	\QuestionAnswers
	{
		\answer any vector in $\Reals^{n}$ can be expressed as a linear combination of $v_{1}, \ldots, v_{k}$
		\answer $v_{1}, \ldots, v_{k}$ are a basis for $\Reals^{n}$
		\answer all the above
		\correctanswer none of the above
		\answer I do not know
	}
	\QuestionSolution{}
	\QuestionAuthor{}
	\QuestionVersion{}
\end{frame}


\begin{frame}{Question E1.18}
	\QuestionKCs{symmetric matrices}
	\QuestionKCsTaxonomies{(1,1)}
	\QuestionBody{Assume both the matrices $A$ and $B$ to be symmetric. Then their product $AB$ is also symmetric.}
	\QuestionAnswers
	{
		\answer true, if and only if $A$ and $B$ are diagonal
		\answer true, always
		\answer false, never
		\correctanswer none of the above
		\answer I do not know
	}
	\QuestionSolution{}
	\QuestionAuthor{}
	\QuestionVersion{}
\end{frame}


\begin{frame}{Question E1.19}
	\QuestionKCs{invertible matrices}
	\QuestionKCsTaxonomies{(1,1)}
	\QuestionBody{Assume that a matrix $A^{2}$ is invertible. Then also $A$ is invertible.}
	\QuestionAnswers
	{
		\answer true, if and only if $A$ is diagonal
		\correctanswer true, always
		\answer false, never
		\answer none of the above
		\answer I do not know
	}
	\QuestionSolution{}
	\QuestionAuthor{}
	\QuestionVersion{}
\end{frame}


\begin{frame}{Question E1.20}
	\QuestionKCs{rotation matrices}
	\QuestionKCsTaxonomies{(1,1)}
	\QuestionBody
	{
		Let
		$$
		A(\theta)
		=
		\begin{bmatrix}
			\cos(\theta) & -\sin(\theta) \\
			\sin(\theta) & \cos(\theta) \\
		\end{bmatrix} .
		$$
	}
	\QuestionAnswers
	{
		\answer $A(\theta_{1}) A(\theta_{2}) = A\left( \theta_{1} \theta_{2} \right)$
		\correctanswer $A(\theta_{1}) A(\theta_{2}) = A\left( \theta_{1} + \theta_{2} \right)$
		\answer $A(\theta_{1}) A(\theta_{2}) = A \left( \theta_{1} \right) + A \left( \theta_{2} \right)$
		\answer none of the above
		\answer I do not know
	}
	\QuestionSolution{}
	\QuestionAuthor{}
	\QuestionVersion{}
\end{frame}


\begin{frame}{Question E1.21}
	\QuestionKCs{determinants}
	\QuestionKCsTaxonomies{(1,1)}
	\QuestionBody
	{
		Let
		$$
		N
		=
		\begin{bmatrix}
			0 & 1 \\
			1 & 0 \\
		\end{bmatrix}
		\qquad
		\qquad
		A
		=
		\begin{bmatrix}
			a & b \\
			c & d \\
		\end{bmatrix} .
		$$
	}
	\QuestionAnswers
	{
		\answer $\det \left( NA \right) = \det \left( N + A \right)$
		\answer $\det \left( NA \right) = \det \left( N - A \right)$
		\answer $\det \left( NA \right) = \det \left( A \right)$
		\correctanswer $\det \left( NA \right) = \det \left( - A \right)$
		\answer none of the above
		\answer I do not know
	}
	\QuestionSolution{}
	\QuestionAuthor{}
	\QuestionVersion{}
\end{frame}


\begin{frame}{Question E1.22}
	\QuestionKCs{determinants}
	\QuestionKCsTaxonomies{(1,1)}
	\QuestionBody
	{
		Let $A, B \in \Reals^{n \times n}$. Then
		$$
		\det
		\left( 
			\begin{bmatrix}
				A & B \\
				B & A \\
			\end{bmatrix}
		\right)
		=
		\det \left( A + B \right)
		\det \left( A - B \right)
		$$
		since
		$$
		\begin{array}{ll}
			\det
			\left( 
				\begin{bmatrix}
					A & B \\
					B & A \\
				\end{bmatrix}
			\right)
			& =
			\det
			\left( 
				A^{2} - B^{2}
			\right) \\
			& =
			\det
			\left( 
				(A + B)
				(A - B)
			\right) \\
			& =
			\det \left( A + B \right)
			\det \left( A - B \right)
		\end{array}
		$$
	}
	\QuestionAnswers
	{
		\answer true
		\answer false
		\correctanswer none of the above
		\answer I do not know
	}
	\QuestionSolution{}
	\QuestionAuthor{}
	\QuestionVersion{}
\end{frame}


\begin{frame}{Question E1.23}
	\QuestionKCs{subspaces}
	\QuestionKCsTaxonomies{(1,1)}
	\QuestionBody{Let $V, W$ be subspaces of the Euclidean vector space $\Reals^{n}$, with bases respectively $\left\{ v_{1}, \ldots, v_{a} \right\}$ and $\left\{ w_{1}, \ldots, w_{b} \right\}$. Then the dimension of the subspace $V + W$ is}
	\QuestionAnswers
	{
		\answer $a + b$, always
		\answer $a + b$, if $V \cap W = \emptyset$
		\correctanswer $a + b$, if $V \cap W = \left\{ 0 \right\}$
		\answer none of the above
		\answer I do not know
	}
	\QuestionSolution{}
	\QuestionAuthor{}
	\QuestionVersion{}
\end{frame}


\begin{frame}{Question E1.24}
	\QuestionKCs{matrix inverses}
	\QuestionKCsTaxonomies{(1,1)}
	\QuestionBody{Let $A \in \Reals^{n \times m}, n \neq m$. A left-inverse for $A$ is a matrix $B \in \Reals^{m \times n}$ s.t.\ $A B = I_{n \times n}$. A right-inverse for $A$ is a matrix $B \in \Reals^{m \times n}$ s.t.\ $B A = I_{m \times m}$. Can $A$ admit both a left- and a right-inverse simultaneously?}
	\QuestionAnswers
	{
		\answer yes, always
		\answer yes, depending on $A$
		\correctanswer no, never
		\answer I do not know
	}
	\QuestionSolution{}
	\QuestionAuthor{}
	\QuestionVersion{}
\end{frame}


\begin{frame}{Question E1.25}
	\QuestionKCs{subspaces}
	\QuestionKCsTaxonomies{(1,1)}
	\QuestionBody{Let $V, W$ be subspaces of the Euclidean vector space $\Reals^{n}$, with bases respectively $\left\{ v_{1}, \ldots, v_{a} \right\}$ and $\left\{ w_{1}, \ldots, w_{b} \right\}$. Then the dimension of the subspace $V + W$ is:}
	\QuestionAnswers
	{
		\answer $a + b$, always
		\answer $a + b$, if $V \cap W = \emptyset$
		\correctanswer $a + b$, if $V \cap W = \left\{ 0 \right\}$
		\answer none of the above
		\answer I do not know
	}
	\QuestionSolution{}
	\QuestionAuthor{}
	\QuestionVersion{}
\end{frame}


\pgfplotsset
{
	PlotsStyle/.style	=
	{
		draw			= black!40!white,
		very thick,
		mark			= none,
	},
	AxesStyle/.style	=
	{
		width			= 5.5cm,
		height			= 2.4cm,
		axis x line		= center,
		axis y line		= center,
		xmin			= -0.5,
		xmax			= 4.5,
		ymin			= -1.5,
		ymax			= 2.2,
		xtick			= {0, 1, 2, 3, 4},
		ytick			= {-1, 0, 1, 2},
		xlabel			= {$t$},
		ylabel			= {$u(t)$},
		every axis y label/.style	= {at={(yticklabel cs:1.0)}, xshift = -0.0cm},
	},
}
\begin{frame}{Question E1.26}
\vspace{-0.9cm} 
	\QuestionKCs{LTI systems}
	\QuestionKCsTaxonomies{(1,1)}
	\QuestionBody
	{
		A LTI system is s.t.\ the input
		\tikz
		{
			\begin{axis}[AxesStyle]
				\addplot [PlotsStyle] coordinates {(-1,0) (0,0) (0,1) (1,1) (1,0) (5,0)};
			\end{axis}
		}
		produces the output
		\tikz
		{
			\begin{axis}[AxesStyle, ylabel = {$y(t)$}]
				\addplot [PlotsStyle] coordinates {(-1,0) (0,0) (0,1) (1,1) (1,-1) (2,-1) (2,0) (5,0)};
			\end{axis}
		}.
		What output will correspond to the input
		\tikz
		{
			\begin{axis}[AxesStyle]
				\addplot [PlotsStyle] coordinates {(-1,0) (0,0) (0,1) (2,1) (2,0) (5,0)};
			\end{axis}
		}?
	}
	\QuestionAnswers
	{
		\answer
		\tikz
		{
			\begin{axis}[AxesStyle, ylabel = {$y(t)$}]
				\addplot [PlotsStyle] coordinates {(-1,0) (0,0) (0,1) (2,1) (2,-1) (4,-1) (4,0) (5,0)};
			\end{axis}
		}
		\answer
		\tikz
		{
			\begin{axis}[AxesStyle, ylabel = {$y(t)$}]
				\addplot [PlotsStyle] coordinates {(-1,0) (0,0) (0,1) (2,1) (2,0) (3,0) (3,-1) (4,-1) (4,0) (5,0)};
			\end{axis}
		}
		\correctanswer
		\tikz
		{
			\begin{axis}[AxesStyle, ylabel = {$y(t)$}]
				\addplot [PlotsStyle] coordinates {(-1,0) (0,0) (0,1) (1,1) (1,0) (2,0) (2,-1) (3,-1) (3,0) (5,0)};
			\end{axis}
		}
		\answer I don't know
	}
% 	\input{figure-question-on-linearity}
% 	TODO
	\QuestionSolution{}
	\QuestionAuthor{}
	\QuestionVersion{}
\end{frame}


\begin{frame}{Question E1.27}
	\QuestionKCs{orthonormality,bases}
	\QuestionKCsTaxonomies{(1,1),(1,1)}
	\QuestionBody{
		Let
		$$
			w =
			\left(
				\frac{i}{\sqrt{3}},
				\frac{i}{\sqrt{3}},
				\frac{i}{\sqrt{3}},
			\right)
			\qquad
			x =
			\left(
				- \frac{2 i}{\sqrt{6}},
				\frac{i}{\sqrt{6}},
				\frac{i}{\sqrt{6}},
			\right)
		 $$
		 $$
			y =
			\left(
				\frac{i}{\sqrt{6}},
				\frac{i}{\sqrt{6}},
				- 2 \frac{i}{\sqrt{6}},
			\right)
			\qquad
			z =
			\left(
				0,
				- \frac{i}{\sqrt{2}},
				\frac{i}{\sqrt{2}},
			\right)
			.
		$$
		Which three of these vectors form an orthonormal basis for $\mathds{C}^{3}$?
	}
	\QuestionAnswers
	{
		\correctanswer $w, x, y$
		\answer $w, x, z$
		\answer $w, y, z$
		\answer I do not know
	}
	\QuestionSolution{}
	\QuestionAuthor{}
	\QuestionVersion{}
\end{frame}


\begin{frame}{Question E1.28}
	\QuestionKCs{eigenvalues}
	\QuestionKCsTaxonomies{(1,1)}
	\QuestionBody{Which are the eigenvalues of
	$
		\displaystyle
		A = \begin{bmatrix} 3 & 5 \\ 3 & 1 \end{bmatrix} 
	$?}
	\QuestionAnswers
	{
		\answer $1, -12$
		\correctanswer $-2, 6$
		\answer $-4, 3$
		\answer I do not know
	}
	\QuestionSolution{}
	\QuestionAuthor{}
	\QuestionVersion{}
\end{frame}


\begin{frame}{Question E1.29}
	\QuestionKCs{probability}
	\QuestionKCsTaxonomies{(1,1)}
	\QuestionBody{I have a hat containing 20 balls, 10 red and 10 blue. I draw 10 balls from the hat. I am interested in the event that I draw exactly five red and five blue balls. Do you think that this is more likely if I note the color of each ball I draw and replace it in the hat, or if I don't replace the balls in the hat after drawing?}
	\QuestionAnswers
	{
		\correctanswer More likely with replacement
		\answer More likely without replacement
		\answer I do not know
	}
	\QuestionSolution{}
	\QuestionAuthor{}
	\QuestionVersion{}
\end{frame}


