% MANUAL: HOW TO WRITE YOUR QUESTIONS IN A DATABASE-FRIENDLY MANNER
%
% Current version: 0.12
%
% --------------------------------------------------------------------
% step 1: surround individual question within
% \begin{IndexedQuestion}
% ...
% \end{IndexedQuestion}
%
% --------------------------------------------------------------------
% step 2: add the information within the IndexedQuestion environment.
% More specifically, add
% \QuestionContentUnits{ZZZ}
% \QuestionTaxonomyLevels{ZZZ}
% For the last two items, check the provided manuals on how to define
% the content units and the taxonomy levels (see the link below these
% instructions)
%
% --------------------------------------------------------------------
% step 3: set the type of the question by adding and setting:
% \QuestionType{ZZZ}
% The potential values for ZZZ above are, for now:
%     multiple choice    open     numeric
%
% --------------------------------------------------------------------
% step 4: add the body of the question within
% \QuestionBody{ZZZ}
% Here you can use plain .tex commands. However DO NOT include figures
% here; in case you need, use the associated field. Also DO NOT use
% self-defined commands! Plain LaTeX, please
%
% --------------------------------------------------------------------
% step 5: ONLY IF NEEDED, add an image through
% \QuestionBodyImage{ZZZ}
% (or \QuestionBodyImage[float]{ZZZ}, with float between 0 and 1.0)
% where ZZZ will be the path of the image you want to show and the
% optional float, if included, specifying the width of the inserted
% image. If you do not have images, it is kind of meaningful to do not
% insert this code (even if inserting a "\QuestionBodyImage{ZZZ}" won't
% produce any output)
%
% --------------------------------------------------------------------
% step 6.a - only for ``multiple choice'' questions: add the potential
% answers through adding
% \QuestionPotentialAnswers{ \answer ZZZ1 \answer \ZZZ2
% \correctanswer ZZZ3 ... }
% Note that you can put as many ``\answer'' and as many
% ``\correctanswer'' as you wish. This field MUST be filled if you
% have a ``multiple choice'' question
%
% --------------------------------------------------------------------
% step 6.b - for the other types of questions: if you want, add the
% potential
% \QuestionSolutions{ZZZ}
% fields, where the ZZZ in the ``correct answer'' field should be the 
% short final answer that likely the students should give at the end 
% of the exercises, while the ``solutions'' field should contain the 
% procedure used to arrive at computing the solution.
% One may also specify the solutions using the field
% \QuestionSolutionsImage[optional]{image-path}
% The optional parameter defines the width of the image.
% These fields are FACULTATIVE
%
% --------------------------------------------------------------------
% step 7: if desired, add information the 'disclosability' of the
% question, by adding and setting opportunely the fields:
% \QuestionDisclosability{ZZZ}
% \QuestionSolutionsDisclosability{ZZZ}
% The potential values for ZZZ above are, for now:
%      only me       only teachers         everybody
% If you do not specify, the default value will be 'everybody'
%
% --------------------------------------------------------------------
% step 8: if you feel like, provide also some additional information
% that you may want to share with either the colleagues or the students
% by adding the remaining fields, opportunely filled. For example
% \QuestionNotesForTheTeachers{ZZZ}
% \QuestionNotesForTheStudents{ZZZ}
% \QuestionFeedbackForTheStudents{ZZZ}
% ... and all the other remaining ones
% Note that it is NOT necessary to fill up these fields


% URL of the current manual on how to define the taxonomy levels (with examples)
% https://docs.google.com/document/d/1En-vn78G7Dcs0IaeFfGd0vht1_ISet01p9Nt-uszOTo/edit


% versions logs
% 
% REMEMBER TO SYNCRONIZE ALSO:
% - "class Question(Document)" in backend/models/models.py
% - "def tex_string_to_question(tex_frame_contents)" in backend/upload/upload_script.py
%
% 0.1  (2020.09.07) -> first publicly released version
% 0.2  (2020.09.19) -> added a field in the questions and added the taxonomy levels in the contents map 
% 0.3  (2020.09.24) -> added the field 'correct answer' in the questions
% 0.4  (2020.09.26) -> added a brief manual
% 0.5  (2020.10.07) -> added a "language" field
% 0.6  (2020.10.11) -> added some links and an empty template
% 0.7  (2021.03.09) -> added a few fields to the questions
% 0.8  (2021.03.17) -> removed a bug on the ``XImages'' fields
% 0.9  (2021.03.22) -> added an option so that one may, if desired, specify the width of the images. Changed QuestionSolutionDisclosability into QuestionSolutionsDisclosability
% 0.10 (2021.03.25) -> removed the QuestionCorrectAnswer field, since obsolete, and changed QuestionImage in QuestionBodyImage
% 0.11 (2021.03.29) -> added the possibility of having pre-defined a \LessonNumber command so to individualize the appearance of the QuestionsCounter index
% 0.12 (2021.03.29) -> modified the behavior of 'AuthorsEmails'


% counter for the environment "indexed questions"
\newcounter		{QuestionsCounter}
\setcounter		{QuestionsCounter}	{0}


\makeatletter
	\@ifpackageloaded{beamerbasemodes}
	{
		% definition when compiling presentations 
		\newenvironment*{IndexedQuestion}	[0]
		{
			\stepcounter{QuestionsCounter}
			\begin{frame}[t]
			{
    			\ifx\LessonNumber\undefined
    			    Question \theQuestionsCounter
    			\else
                    Question \LessonNumber.\theQuestionsCounter
                \fi
            }
			\begingroup
		}
		{
			\endgroup
			\end{frame}
		}
	}
	{
		% definition when compiling A4-like documents
		\newenvironment{IndexedQuestion}
		{
			\stepcounter{QuestionsCounter}
			\begin{center}
			\begin{tabular}{|p{0.9\textwidth}|}
			\hline
			\cellcolor{white!90!black} % requires \usepackage[table]{xcolor}
			\ifx\LessonNumber\undefined
			    \emph{\textbf{Question \theQuestionsCounter}} \vspace{0.1cm} \\
			\else
                \emph{\textbf{Question \LessonNumber.\theQuestionsCounter}} \vspace{0.1cm} \\
            \fi
			\\
		}
		{
			\\
			\hline
			\end{tabular} 
			\end{center}
		}
	}
\makeatother


% commands for indexing the contents
\newcommand {\QuestionLanguage}	    			[1]	{} % specify only whether this is different from English
\newcommand {\QuestionContentUnits}				[1]	{} % comma separated list
\newcommand {\QuestionTaxonomyLevels}			[1]	{} % comma separated list, must be as many as the CUs
\newcommand {\QuestionType}						[1]	{} % [ multiple choice | open | numeric ]
\newcommand {\QuestionBody}						[1]	{#1} % in plain LaTeX -- avoid using self-defined commands
\newcommand {\QuestionBodyImage}				[2][]  % better to keep commented if unused
{
   	\ifx#2\empty
   	    % do nothing
   	\else
     	\ifx#1\empty
    		$\;$ \\
    		\includegraphics[width = 0.9\columnwidth]{#2}
    	\else
    		$\;$ \\
    		\includegraphics[width = #1\columnwidth]{#2}
    	\fi
	\fi
}
\newcommand {\QuestionPotentialAnswers}			[1]	{\begin{enumerate} #1 \end{enumerate}} % meant to be used in the case ``multiple choice''. Will include a series of ``\answer'' and ``\correctanswer''
\newcommand {\QuestionSolutions}				[1] % meant to be used in the cases ``open question'' and ``numerical question'' for a long description of *how* to solve the question. May also be an URL to an external resource
{
	\ifshowsolutions
		\ifx#1\empty
			% do nothing
		\else
			\textbf{\underline{Solution:}} #1
		\fi
	\fi
}													   % useful especially for open questions
\newcommand {\QuestionSolutionsImage}			[2][]  % better to keep commented if unused
{
   	\ifx#2\empty
   	    % do nothing
   	\else
     	\ifx#1\empty
    		$\;$ \\
    		\includegraphics[width = 0.9\columnwidth]{#2}
    	\else
    		$\;$ \\
    		\includegraphics[width = #1\columnwidth]{#2}
    	\fi
	\fi
}
\newcommand {\QuestionNotesForTheTeachers}		[1]	{} % in case you want to give some additional information to the teachers
\newcommand {\QuestionNotesForTheStudents}		[1]	{} % in case you want to give some additional information to the students
\newcommand {\QuestionFeedbackForTheStudents}	[1]	{} % text that should be displayed to the students after they have answered to the question, useful especially when using the questions in a LMS
\newcommand {\QuestionDisclosability}			[1]	{} % [ only me, only teachers, only my institution, everybody ]
\newcommand {\QuestionSolutionsDisclosability}	[1]	{} % [ only me, only teachers, only my institution, everybody ]
\newcommand {\QuestionNotationStandard}     	[1]	{} % to indicate which notation standard is used. E.g., ABC vs FGH for state space models; see the manuals in the portal for an updated list of notation standards

% only useful for when one makes "multiple choice" questions
\newcommand {\answer}							[0]	{\item}
\newcommand {\correctanswer}					[0]	{\item \ifshowsolutions \textbf{(\underline{correct})} \fi }

% fields that should NOT be compiled by the users
\newcommand	{\QuestionAuthorsEmails}			[1]	{} % do not fill up by yourself - it will be compiled by the database
\newcommand {\QuestionID}						[1]	{} % do not fill up by yourself - it will be compiled by the database

%%% definition of the environment "content maps" and its ancillary ones %%%

\makeatletter
	\@ifpackageloaded{beamerbasemodes}
	{
		% definition when compiling presentations 
		\newenvironment*{ContentsMap}	[0]
		{
			\begin{frame}[t]{Contents map}
			\begingroup
		}
		{
			\endgroup
			\end{frame}
		}
	}
	{
		% definition when compiling normal documents 
		\newenvironment*{ContentsMap}	[0]
		{
			\section*{Contents map}
			\begingroup
		}
		{
			\endgroup
		}
	}
\makeatother

\newenvironment{DevelopedContents}
{
	\begin{center}
	\rowcolors
	{2}					% index of the first row to be colored
	{white!95!black}	% color of the first colored row (and the third, fifth, etc)
	{}					% color of the second colored row (and the fourth, sixth, etc)
	\begin{tabular}{|p{0.6\textwidth}p{0.3\textwidth}|}
	\hline
	\cellcolor{white!90!black} % requires \usepackage[table]{xcolor}
	\emph{\textbf{developed content units}} &
	\cellcolor{white!90!black} % requires \usepackage[table]{xcolor}
	\emph{\textbf{taxonomy levels}} \\
}
{ 
	\hline
	\end{tabular} 
	\end{center}
}

\newenvironment{PrerequisiteContents}
{
	\begin{center}
	\rowcolors
	{2}					% index of the first row to be colored
	{white!95!black}	% color of the first colored row (and the third, fifth, etc)
	{}					% color of the second colored row (and the fourth, sixth, etc)
	\begin{tabular}{|p{0.6\textwidth}p{0.3\textwidth}|}
	\hline
	\cellcolor{white!90!black} % requires \usepackage[table]{xcolor}
	\emph{\textbf{prerequisite content units}} &
	\cellcolor{white!90!black} % requires \usepackage[table]{xcolor}
	\emph{\textbf{taxonomy levels}} \\
}
{ 
	\hline
	\end{tabular} 
	\end{center}
}

\makeatletter
	\@ifpackageloaded{beamerbasemodes}
	{
		% definition when compiling presentations 
		\newenvironment*{ContentsRelationships}	[0]
		{
			\begin{block}{Contents relationships}
			\begingroup
		}
		{
			\endgroup
			\end{block}
		}
	}
	{
		% definition when compiling normal documents 
		\newenvironment*{ContentsRelationships}	[0]
		{
			\subsection*{Contents relationships}
			\begingroup
		}
		{
			\endgroup
		}
	}
\makeatother

\newenvironment{WhatIsNecessaryForWhatRelationships}
{
	\begin{center}
	\rowcolors
	{2}					% index of the first row to be colored
	{white!95!black}	% color of the first colored row (and the third, fifth, etc)
	{}					% color of the second colored row (and the fourth, sixth, etc)
	\begin{tabular}{|p{0.35\textwidth}p{0.10\textwidth}|p{0.35\textwidth}p{0.10\textwidth}|}
	\hline
	\multicolumn{4}{|c|}
	{
		\cellcolor{white!90!black} % requires \usepackage[table]{xcolor}
		\emph{\textbf{``to learn $x$ at t.l.\ $\alpha$ it is necessary to know $y$ at t.l.\ $\beta$'' relations}}
	} \\
}
{ 
	\hline
	\end{tabular} 
	\end{center}
}

\newenvironment{WhatIsUsefulForWhatRelationships}
{
	\begin{center}
	\rowcolors
	{2}					% index of the first row to be colored
	{white!95!black}	% color of the first colored row (and the third, fifth, etc)
	{}					% color of the second colored row (and the fourth, sixth, etc)
	\begin{tabular}{|p{0.35\textwidth}p{0.10\textwidth}|p{0.35\textwidth}p{0.10\textwidth}|}
	\hline
	\multicolumn{4}{|c|}
	{
		\cellcolor{white!90!black} % requires \usepackage[table]{xcolor}
		\emph{\textbf{``to learn $x$ at t.l.\ $\alpha$ it is useful to know $y$ at t.l.\ $\beta$'' relations}}
	} \\
}
{ 
	\hline
	\end{tabular} 
	\end{center}
}

\newenvironment{WhatGeneralizesWhatRelationships}
{
	\begin{center}
	\rowcolors
	{2}					% index of the first row to be colored
	{white!95!black}	% color of the first colored row (and the third, fifth, etc)
	{}					% color of the second colored row (and the fourth, sixth, etc)
	\begin{tabular}{|p{0.45\textwidth}|p{0.45\textwidth}|}
	\hline
	\multicolumn{2}{|c|}
	{
		\cellcolor{white!90!black} % requires \usepackage[table]{xcolor}
		\emph{\textbf{``$x$ is generalized by / contained in $y$'' relations}}
	} \\
}
{ 
	\hline
	\end{tabular} 
	\end{center}
}

\newenvironment{WhatIsASynonymOfWhatRelationships}
{
	\begin{center}
	\rowcolors
	{2}					% index of the first row to be colored
	{white!95!black}	% color of the first colored row (and the third, fifth, etc)
	{}					% color of the second colored row (and the fourth, sixth, etc)
	\begin{tabular}{|p{0.45\textwidth}|p{0.45\textwidth}|}
	\hline
	\multicolumn{2}{|c|}
	{
		\cellcolor{white!90!black} % requires \usepackage[table]{xcolor}
		\emph{\textbf{``$x$ is a synonym of $y$'' relations}}
	} \\
}
{ 
	\hline
	\end{tabular} 
	\end{center}
}

\newenvironment{WhatIsDirectlyLogicallyConnectedToWhatRelationships}
{
	\begin{center}
	\rowcolors
	{2}					% index of the first row to be colored
	{white!95!black}	% color of the first colored row (and the third, fifth, etc)
	{}					% color of the second colored row (and the fourth, sixth, etc)
	\begin{tabular}{|p{0.35\textwidth}p{0.10\textwidth}|p{0.35\textwidth}p{0.10\textwidth}|}
	\hline
	\multicolumn{4}{|c|}
	{
		\cellcolor{white!90!black} % requires \usepackage[table]{xcolor}
		\emph{\textbf{``$x$ at t.l.\ $\alpha$ is directly logically connected to $y$ at t.l.\ $\beta$'' relations}}
	} \\
}
{ 
	\hline
	\end{tabular} 
	\end{center}
}


% % EMPTY TEMPLATE
% %
% % Note: please delete the fields that you are not using 
% %
% \begin{IndexedQuestion}
% %
% % fields that MUST be filled up
% \QuestionContentUnits{FILL ME} % comma separated list
% \QuestionTaxonomyLevels{FILL ME} % comma separated list, must be as many as the CUs
% \QuestionType{FILL ME} % options: [ multiple choice | open | numeric ]
% \QuestionBody{FILL ME} % use plain latex code, and try to avoid user-defined commands
% %
% % field that must be filled up only with multiple choice questions.
% You can obviously change the order of correct vs wrong answers,
% have as many answers as you want, and add as many correct answers as you want
% \QuestionPotentialAnswers{ \answer BLABLA1 \correctanswer BLABLA2 \answer BLABLA3 ... }
% %
% % fields that you may also skip filling up
% \QuestionLanguage{FILL ME} % if not provided, then this means ``English''
% \QuestionBodyImage[OPTIONAL WIDTH]{FILL ME}
% \QuestionSolutions{FILL ME} % meant to be used in the cases ``open question'' and ``numerical question'' for a long description of the correct answer and ideally *how* to solve the question. May also be a URL
% \QuestionSolutionsImage[OPTIONAL WIDTH]{FILL ME} % alternative way of providing a solution
% \QuestionNotesForTheTeachers{FILL ME} % some additional notes, in case you want to share them
% \QuestionNotesForTheStudents{FILL ME} % some additional notes, in case you want to share them
% \QuestionFeedbackForTheStudents{FILL ME} % useful especially when using the questions in a LMS
% \QuestionDisclosability{FILL ME} % options: [ only me, only teachers, only my institution, everybody ]. If not provided, then this means ``everybody''
% \QuestionSolutionsDisclosability{FILL ME} % options: [ only me, only teachers, only my institution, everybody ]. If not provided, then this means ``everybody''
% \QuestionNotationStandard{FILL ME} % see the manuals in the portal for an updated list of notation standards
% %
% \end{IndexedQuestion}
%
% Note: if you have defined somewhere the command \LessonNumber then the .pdf
% will display ``Question \LessonNumber.\QuestionNumber''. This is useful to
% have question codes that are unique across different lessons in a course

% \begin{IndexedQuestion}
% 	\QuestionContentUnits{<++>}
% 	\QuestionTaxonomyLevels{<++>}
% 	\QuestionType{<++>}
% 	\QuestionBody{<++>}
% 	\QuestionPotentialAnswers{<++>}
% \end{IndexedQuestion}

