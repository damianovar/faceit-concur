<<<<<<< HEAD
% Current version: 0.2
%
% versions logs
% 0.1 (2020.09.07) -> first publicly released version
% 0.2 (2020.09.19) -> added a field in the questions and added the taxonomy levels in the contents map 


% counter for the environment "indexed questions"
\newcounter		{QuestionsCounter}
\setcounter		{QuestionsCounter}	{0}


\makeatletter
	\@ifpackageloaded{beamerbasemodes}
	{
		% definition when compiling presentations 
		\newenvironment*{IndexedQuestion}	[0]
		{
			\stepcounter{QuestionsCounter}
			\begin{frame}[t]{Question \theQuestionsCounter}
			\begingroup
		}
		{
			\endgroup
			\end{frame}
		}
	}
	{
		% definition when compiling A4-like documents
		\newenvironment{IndexedQuestion}
		{
			\stepcounter{QuestionsCounter}
			\begin{center}
			\begin{tabular}{|p{0.9\textwidth}|}
			\hline
			\cellcolor{white!90!black} % requires \usepackage[table]{xcolor}
			\emph{\textbf{Question \theQuestionsCounter}} \vspace{0.1cm} \\
			\\
		}
		{ 
			\\
			\hline
			\end{tabular} 
			\end{center}
		}
	}
\makeatother


% commands for indexing the contents
\newcommand	{\QuestionAuthorEmail}				[1]	{}
\newcommand {\QuestionID}						[1]	{} % do not fill up by yourself - it will be compiled by the database
\newcommand {\QuestionContentUnits}				[1]	{} % comma separated list
\newcommand {\QuestionTaxonomyLevels}			[1]	{} % comma separated list, must be as many as the CUs
\newcommand {\QuestionType}						[1]	{} % [ multiple choice | open | numeric ]
\newcommand {\QuestionBody}						[1]	{#1}
\newcommand {\QuestionImage}					[1]	   % keep commented if unused
{
	\ifx#1\empty
		% do nothing
	\else
		\begin{figure}[!htbp]
			\centering
			\includegraphics[width = 0.9\columnwidth]{#1}
			\caption{Figure relative to Question~\theQuestionsCounter.}
		\end{figure}
	\fi
}
\newcommand {\QuestionPotentialAnswers}			[1]	{\begin{enumerate} #1 \end{enumerate}}
\newcommand {\QuestionSolutions}				[1]
{
	\ifshowsolutions
		\ifx#1\empty
			% do nothing
		\else
			Solution: #1
		\fi
	\fi
}													   % useful especially for open questions
\newcommand {\QuestionNotesForTheTeachers}		[1]	{}
\newcommand {\QuestionNotesForTheStudents}		[1]	{}
\newcommand {\QuestionFeedbackForTheStudents}	[1]	{} % useful especially when using the questions in a LMS
\newcommand {\QuestionDisclosability}			[1]	{} % [ only me, only teachers, everybody ]
\newcommand {\QuestionSolutionDisclosability}	[1]	{} % [ only me, only teachers, everybody ]
\newcommand {\answer}							[0]	{\item}
\newcommand {\correctanswer}					[0]	{\item \ifshowsolutions (correct) \fi }


%%% definition of the environment "content maps" and its ancillary ones %%%

\makeatletter
	\@ifpackageloaded{beamerbasemodes}
	{
		% definition when compiling presentations 
		\newenvironment*{ContentsMap}	[0]
		{
			\begin{frame}[t]{Contents map}
			\begingroup
		}
		{
			\endgroup
			\end{frame}
		}
	}
	{
		% definition when compiling normal documents 
		\newenvironment*{ContentsMap}	[0]
		{
			\section*{Contents map}
			\begingroup
		}
		{
			\endgroup
		}
	}
\makeatother

\newenvironment{DevelopedContents}
{
	\begin{center}
	\rowcolors
	{2}					% index of the first row to be colored
	{white!95!black}	% color of the first colored row (and the third, fifth, etc)
	{}					% color of the second colored row (and the fourth, sixth, etc)
	\begin{tabular}{|p{0.6\textwidth}p{0.3\textwidth}|}
	\hline
	\cellcolor{white!90!black} % requires \usepackage[table]{xcolor}
	\emph{\textbf{developed content units}} &
	\cellcolor{white!90!black} % requires \usepackage[table]{xcolor}
	\emph{\textbf{taxonomy levels}} \\
}
{ 
	\hline
	\end{tabular} 
	\end{center}
}

\newenvironment{PrerequisiteContents}
{
	\begin{center}
	\rowcolors
	{2}					% index of the first row to be colored
	{white!95!black}	% color of the first colored row (and the third, fifth, etc)
	{}					% color of the second colored row (and the fourth, sixth, etc)
	\begin{tabular}{|p{0.6\textwidth}p{0.3\textwidth}|}
	\hline
	\cellcolor{white!90!black} % requires \usepackage[table]{xcolor}
	\emph{\textbf{prerequisite content units}} &
	\cellcolor{white!90!black} % requires \usepackage[table]{xcolor}
	\emph{\textbf{taxonomy levels}} \\
}
{ 
	\hline
	\end{tabular} 
	\end{center}
}

\makeatletter
	\@ifpackageloaded{beamerbasemodes}
	{
		% definition when compiling presentations 
		\newenvironment*{ContentsRelationships}	[0]
		{
			\begin{block}{Contents relationships}
			\begingroup
		}
		{
			\endgroup
			\end{block}
		}
	}
	{
		% definition when compiling normal documents 
		\newenvironment*{ContentsRelationships}	[0]
		{
			\subsection*{Contents relationships}
			\begingroup
		}
		{
			\endgroup
		}
	}
\makeatother

\newenvironment{WhatIsNecessaryForWhatRelationships}
{
	\begin{center}
	\rowcolors
	{2}					% index of the first row to be colored
	{white!95!black}	% color of the first colored row (and the third, fifth, etc)
	{}					% color of the second colored row (and the fourth, sixth, etc)
	\begin{tabular}{|p{0.35\textwidth}p{0.10\textwidth}|p{0.35\textwidth}p{0.10\textwidth}|}
	\hline
	\multicolumn{4}{|c|}
	{
		\cellcolor{white!90!black} % requires \usepackage[table]{xcolor}
		\emph{\textbf{``to learn $x$ at t.l.\ $\alpha$ it is necessary to know $y$ at t.l.\ $\beta$'' relations}}
	} \\
}
{ 
	\hline
	\end{tabular} 
	\end{center}
}

\newenvironment{WhatIsUsefulForWhatRelationships}
{
	\begin{center}
	\rowcolors
	{2}					% index of the first row to be colored
	{white!95!black}	% color of the first colored row (and the third, fifth, etc)
	{}					% color of the second colored row (and the fourth, sixth, etc)
	\begin{tabular}{|p{0.35\textwidth}p{0.10\textwidth}|p{0.35\textwidth}p{0.10\textwidth}|}
	\hline
	\multicolumn{4}{|c|}
	{
		\cellcolor{white!90!black} % requires \usepackage[table]{xcolor}
		\emph{\textbf{``to learn $x$ at t.l.\ $\alpha$ it is useful to know $y$ at t.l.\ $\beta$'' relations}}
	} \\
}
{ 
	\hline
	\end{tabular} 
	\end{center}
}

\newenvironment{WhatGeneralizesWhatRelationships}
{
	\begin{center}
	\rowcolors
	{2}					% index of the first row to be colored
	{white!95!black}	% color of the first colored row (and the third, fifth, etc)
	{}					% color of the second colored row (and the fourth, sixth, etc)
	\begin{tabular}{|p{0.45\textwidth}|p{0.45\textwidth}|}
	\hline
	\multicolumn{2}{|c|}
	{
		\cellcolor{white!90!black} % requires \usepackage[table]{xcolor}
		\emph{\textbf{``$x$ is generalized by / contained in $y$'' relations}}
	} \\
}
{ 
	\hline
	\end{tabular} 
	\end{center}
}

\newenvironment{WhatIsASynonymOfWhatRelationships}
{
	\begin{center}
	\rowcolors
	{2}					% index of the first row to be colored
	{white!95!black}	% color of the first colored row (and the third, fifth, etc)
	{}					% color of the second colored row (and the fourth, sixth, etc)
	\begin{tabular}{|p{0.45\textwidth}|p{0.45\textwidth}|}
	\hline
	\multicolumn{2}{|c|}
	{
		\cellcolor{white!90!black} % requires \usepackage[table]{xcolor}
		\emph{\textbf{``$x$ is a synonym of $y$'' relations}}
	} \\
}
{ 
	\hline
	\end{tabular} 
	\end{center}
}

\newenvironment{WhatIsDirectlyLogicallyConnectedToWhatRelationships}
{
	\begin{center}
	\rowcolors
	{2}					% index of the first row to be colored
	{white!95!black}	% color of the first colored row (and the third, fifth, etc)
	{}					% color of the second colored row (and the fourth, sixth, etc)
	\begin{tabular}{|p{0.35\textwidth}p{0.10\textwidth}|p{0.35\textwidth}p{0.10\textwidth}|}
	\hline
	\multicolumn{4}{|c|}
	{
		\cellcolor{white!90!black} % requires \usepackage[table]{xcolor}
		\emph{\textbf{``$x$ at t.l.\ $\alpha$ is directly logically connected to $y$ at t.l.\ $\beta$'' relations}}
	} \\
}
{ 
	\hline
	\end{tabular} 
	\end{center}
}

=======
% Current version: 0.2
%
% versions logs
% 0.1 (2020.09.07) -> first publicly released version
% 0.2 (2020.09.19) -> added a field in the questions and added the taxonomy levels in the contents map 


% counter for the environment "indexed questions"
\newcounter		{QuestionsCounter}
\setcounter		{QuestionsCounter}	{0}


\makeatletter
	\@ifpackageloaded{beamerbasemodes}
	{
		% definition when compiling presentations 
		\newenvironment*{IndexedQuestion}	[0]
		{
			\stepcounter{QuestionsCounter}
			\begin{frame}[t]{Question \theQuestionsCounter}
			\begingroup
		}
		{
			\endgroup
			\end{frame}
		}
	}
	{
		% definition when compiling A4-like documents
		\newenvironment{IndexedQuestion}
		{
			\stepcounter{QuestionsCounter}
			\begin{center}
			\begin{tabular}{|p{0.9\textwidth}|}
			\hline
			\cellcolor{white!90!black} % requires \usepackage[table]{xcolor}
			\emph{\textbf{Question \theQuestionsCounter}} \vspace{0.1cm} \\
			\\
		}
		{ 
			\\
			\hline
			\end{tabular} 
			\end{center}
		}
	}
\makeatother


% commands for indexing the contents
\newcommand	{\QuestionAuthorEmail}				[1]	{}
\newcommand {\QuestionID}						[1]	{} % do not fill up by yourself - it will be compiled by the database
\newcommand {\QuestionContentUnits}				[1]	{} % comma separated list
\newcommand {\QuestionTaxonomyLevels}			[1]	{} % comma separated list, must be as many as the CUs
\newcommand {\QuestionType}						[1]	{} % [ multiple choice | open | numeric ]
\newcommand {\QuestionBody}						[1]	{#1}
\newcommand {\QuestionImage}					[1]	   % keep commented if unused
{
	\ifx#1\empty
		% do nothing
	\else
		\begin{figure}[!htbp]
			\centering
			\includegraphics[width = 0.9\columnwidth]{#1}
			\caption{Figure relative to Question~\theQuestionsCounter.}
		\end{figure}
	\fi
}
\newcommand {\QuestionPotentialAnswers}			[1]	{\begin{enumerate} #1 \end{enumerate}}
\newcommand {\QuestionSolutions}				[1]
{
	\ifshowsolutions
		\ifx#1\empty
			% do nothing
		\else
			Solution: #1
		\fi
	\fi
}													   % useful especially for open questions
\newcommand {\QuestionNotesForTheTeachers}		[1]	{}
\newcommand {\QuestionNotesForTheStudents}		[1]	{}
\newcommand {\QuestionFeedbackForTheStudents}	[1]	{} % useful especially when using the questions in a LMS
\newcommand {\QuestionDisclosability}			[1]	{} % [ only me, only teachers, everybody ]
\newcommand {\QuestionSolutionDisclosability}	[1]	{} % [ only me, only teachers, everybody ]
\newcommand {\answer}							[0]	{\item}
\newcommand {\correctanswer}					[0]	{\item \ifshowsolutions (correct) \fi }


%%% definition of the environment "content maps" and its ancillary ones %%%

\makeatletter
	\@ifpackageloaded{beamerbasemodes}
	{
		% definition when compiling presentations 
		\newenvironment*{ContentsMap}	[0]
		{
			\begin{frame}[t]{Contents map}
			\begingroup
		}
		{
			\endgroup
			\end{frame}
		}
	}
	{
		% definition when compiling normal documents 
		\newenvironment*{ContentsMap}	[0]
		{
			\section*{Contents map}
			\begingroup
		}
		{
			\endgroup
		}
	}
\makeatother

\newenvironment{DevelopedContents}
{
	\begin{center}
	\rowcolors
	{2}					% index of the first row to be colored
	{white!95!black}	% color of the first colored row (and the third, fifth, etc)
	{}					% color of the second colored row (and the fourth, sixth, etc)
	\begin{tabular}{|p{0.6\textwidth}p{0.3\textwidth}|}
	\hline
	\cellcolor{white!90!black} % requires \usepackage[table]{xcolor}
	\emph{\textbf{developed content units}} &
	\cellcolor{white!90!black} % requires \usepackage[table]{xcolor}
	\emph{\textbf{taxonomy levels}} \\
}
{ 
	\hline
	\end{tabular} 
	\end{center}
}

\newenvironment{PrerequisiteContents}
{
	\begin{center}
	\rowcolors
	{2}					% index of the first row to be colored
	{white!95!black}	% color of the first colored row (and the third, fifth, etc)
	{}					% color of the second colored row (and the fourth, sixth, etc)
	\begin{tabular}{|p{0.6\textwidth}p{0.3\textwidth}|}
	\hline
	\cellcolor{white!90!black} % requires \usepackage[table]{xcolor}
	\emph{\textbf{prerequisite content units}} &
	\cellcolor{white!90!black} % requires \usepackage[table]{xcolor}
	\emph{\textbf{taxonomy levels}} \\
}
{ 
	\hline
	\end{tabular} 
	\end{center}
}

\makeatletter
	\@ifpackageloaded{beamerbasemodes}
	{
		% definition when compiling presentations 
		\newenvironment*{ContentsRelationships}	[0]
		{
			\begin{block}{Contents relationships}
			\begingroup
		}
		{
			\endgroup
			\end{block}
		}
	}
	{
		% definition when compiling normal documents 
		\newenvironment*{ContentsRelationships}	[0]
		{
			\subsection*{Contents relationships}
			\begingroup
		}
		{
			\endgroup
		}
	}
\makeatother

\newenvironment{WhatIsNecessaryForWhatRelationships}
{
	\begin{center}
	\rowcolors
	{2}					% index of the first row to be colored
	{white!95!black}	% color of the first colored row (and the third, fifth, etc)
	{}					% color of the second colored row (and the fourth, sixth, etc)
	\begin{tabular}{|p{0.35\textwidth}p{0.10\textwidth}|p{0.35\textwidth}p{0.10\textwidth}|}
	\hline
	\multicolumn{4}{|c|}
	{
		\cellcolor{white!90!black} % requires \usepackage[table]{xcolor}
		\emph{\textbf{``to learn $x$ at t.l.\ $\alpha$ it is necessary to know $y$ at t.l.\ $\beta$'' relations}}
	} \\
}
{ 
	\hline
	\end{tabular} 
	\end{center}
}

\newenvironment{WhatIsUsefulForWhatRelationships}
{
	\begin{center}
	\rowcolors
	{2}					% index of the first row to be colored
	{white!95!black}	% color of the first colored row (and the third, fifth, etc)
	{}					% color of the second colored row (and the fourth, sixth, etc)
	\begin{tabular}{|p{0.35\textwidth}p{0.10\textwidth}|p{0.35\textwidth}p{0.10\textwidth}|}
	\hline
	\multicolumn{4}{|c|}
	{
		\cellcolor{white!90!black} % requires \usepackage[table]{xcolor}
		\emph{\textbf{``to learn $x$ at t.l.\ $\alpha$ it is useful to know $y$ at t.l.\ $\beta$'' relations}}
	} \\
}
{ 
	\hline
	\end{tabular} 
	\end{center}
}

\newenvironment{WhatGeneralizesWhatRelationships}
{
	\begin{center}
	\rowcolors
	{2}					% index of the first row to be colored
	{white!95!black}	% color of the first colored row (and the third, fifth, etc)
	{}					% color of the second colored row (and the fourth, sixth, etc)
	\begin{tabular}{|p{0.45\textwidth}|p{0.45\textwidth}|}
	\hline
	\multicolumn{2}{|c|}
	{
		\cellcolor{white!90!black} % requires \usepackage[table]{xcolor}
		\emph{\textbf{``$x$ is generalized by / contained in $y$'' relations}}
	} \\
}
{ 
	\hline
	\end{tabular} 
	\end{center}
}

\newenvironment{WhatIsASynonymOfWhatRelationships}
{
	\begin{center}
	\rowcolors
	{2}					% index of the first row to be colored
	{white!95!black}	% color of the first colored row (and the third, fifth, etc)
	{}					% color of the second colored row (and the fourth, sixth, etc)
	\begin{tabular}{|p{0.45\textwidth}|p{0.45\textwidth}|}
	\hline
	\multicolumn{2}{|c|}
	{
		\cellcolor{white!90!black} % requires \usepackage[table]{xcolor}
		\emph{\textbf{``$x$ is a synonym of $y$'' relations}}
	} \\
}
{ 
	\hline
	\end{tabular} 
	\end{center}
}

\newenvironment{WhatIsDirectlyLogicallyConnectedToWhatRelationships}
{
	\begin{center}
	\rowcolors
	{2}					% index of the first row to be colored
	{white!95!black}	% color of the first colored row (and the third, fifth, etc)
	{}					% color of the second colored row (and the fourth, sixth, etc)
	\begin{tabular}{|p{0.35\textwidth}p{0.10\textwidth}|p{0.35\textwidth}p{0.10\textwidth}|}
	\hline
	\multicolumn{4}{|c|}
	{
		\cellcolor{white!90!black} % requires \usepackage[table]{xcolor}
		\emph{\textbf{``$x$ at t.l.\ $\alpha$ is directly logically connected to $y$ at t.l.\ $\beta$'' relations}}
	} \\
}
{ 
	\hline
	\end{tabular} 
	\end{center}
}

>>>>>>> 2d94842926ecc88e384ed8f7e31e15586e81dca7
