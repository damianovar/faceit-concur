\begin{frame}{Question E6.1}
	\QuestionKCs{projection matrices}
	\QuestionKCsTaxonomies{(1,1)}
	\QuestionBody{a linear transformation $P$ from a vector space to itself such that $P^{2}=P$ is a\ldots}
	\QuestionAnswers
	{
		\answer linear map
		\answer continuous map
		\correctanswer projection
		\answer I do not know
	}
	\QuestionSolution{this is the definition of projection}
\end{frame}


\begin{frame}{Question E6.2}
	\QuestionKCs{projection matrices}
	\QuestionKCsTaxonomies{(1,1)}
	\QuestionBody{is $P = \begin{bmatrix} 0 & 0 \\ \alpha & 1 \end{bmatrix}$ a projection matrix?}
	\QuestionAnswers
	{
		\correctanswer yes
		\answer no
		\answer I do not know
	}
	\QuestionSolution{one should do the verification and see whether $P^{2} = P$}
\end{frame}


\begin{frame}{Question E6.3}
	\QuestionKCs{orthogonal projection matrices}
	\QuestionKCsTaxonomies{(1,1)}
	\QuestionBody{when is $P = \begin{bmatrix} 0 & 0 \\ \alpha & 1 \end{bmatrix}$ an orthogonal projection matrix?}
	\QuestionAnswers
	{
		\correctanswer when $\alpha = 0$
		\answer when $\alpha = 1$
		\answer I do not know
	}
	\QuestionSolution{a projection is orthogonal if $P = P^{T} = P^{2}$, i.e., if $P$ is symmetric (Hermitian in case we deal with complex $P$s). In this case the solution is $\alpha = 0$}
\end{frame}


\begin{frame}{Question E6.4}
	\QuestionKCs{scores}
	\QuestionKCsTaxonomies{(1,1)}
	\QuestionBody
	{
		Which vectors are the score vectors? \vspace{0.2cm} \\
		\centering
		\includegraphics[width = 0.8\textwidth]{PCA_vectorform}
	}
	\QuestionAnswers
	{
		\answer the $p$'s
		\correctanswer the $t$'s
		\answer I do not know
	}
	\QuestionSolution{the loadings have the same dimensionality of each sample (i.e., if there are 3 variables, then there are 3 loadings). The scores have the same dimensionality of the \emph{number} of samples}
\end{frame}


\begin{frame}{Question E6.5}
	\QuestionKCs{loadings}
	\QuestionKCsTaxonomies{(1,1)}
	\QuestionBody
	{
		Which vectors are the loading vectors? \vspace{0.2cm} \\
		\centering
		\includegraphics[width = 0.8\textwidth]{PCA_vectorform}
	}
	\QuestionAnswers
	{
		\correctanswer the $p$'s
		\answer the $t$'s
		\answer I do not know
	}
	\QuestionSolution{the loadings have the same dimensionality of each sample (i.e., if there are 3 variables, then there are 3 loadings). The scores have the same dimensionality of the \emph{number} of samples}
\end{frame}


\begin{frame}{Question E6.6}
	\QuestionKCs{scores plot}
	\QuestionKCsTaxonomies{(1,1)}
	\QuestionBody
	{
		Each sample in the dataset plotted in the left represents a vector of 3 variables. Which type of plot is the one in the right? \vspace{0.2cm} \\
		\centering
		\includegraphics[width = 0.8\textwidth]{score-plot-for-peer-instructions}
	}
	\QuestionAnswers
	{
		\correctanswer a scores plot
		\answer a loadings plot
		\answer a correlation loadings plot
		\answer I do not know
	}
	\QuestionSolution{the loadings have the same dimensionality of each sample (i.e., if there are 3 variables, then there are 3 loadings). The scores have the same dimensionality of the \emph{number} of samples. Since here there are more than 3 samples in the plot, it must be a scores plot}
\end{frame}


\begin{frame}{Question E6.7}
	\QuestionKCs{loadings plot}
	\QuestionKCsTaxonomies{(1,1)}
	\QuestionBody
	{
		Each sample in the dataset plotted in the left represents a vector of 3 variables. If in the right we were plotting the loadings plot, how many samples would we have seen? \vspace{0.2cm} \\
		\centering
		\includegraphics[width = 0.8\textwidth]{score-plot-for-peer-instructions}
	}
	\QuestionAnswers
	{
		\answer a number equal to the number of the samples in the dataset
		\correctanswer 3
		\answer a number equal to the number of PCs we computed
		\answer I do not know
	}
	\QuestionSolution{given the fact that the loadings have the same dimensionality of each sample (i.e., if there are 3 variables, then there are 3 loadings), here we must have 3 dots}
\end{frame}


\begin{frame}{Question E6.8}
	\QuestionKCs{scores}
	\QuestionKCsTaxonomies{(1,1)}
	\QuestionBody
	{
		Consider the scores plot in the right. What do the two scores components highlighted in the scores plot say about the original sample in the dataset? \vspace{0.2cm} \\
		\centering
		\includegraphics[width = 0.8\textwidth]{score-plot-for-peer-instructions}
	}
	\QuestionAnswers
	{
		\answer the relative importance of that two PCs for that specific sample   
		\answer They indicate the distance from the center of the dataset for that specific sample
		\correctanswer They indicate the distance from the center of the dataset along that PCs for that specific sample
		\answer I do not know
	}
	\QuestionSolution{}
\end{frame}


\begin{frame}{Question E6.9}
	\QuestionKCs{principal components}
	\QuestionKCsTaxonomies{(1,1)}
	\QuestionBody
	{
		What determines the direction of the ``PC1'' line? \vspace{0.2cm} \\
		\centering
		\includegraphics[width = 0.4\textwidth]{pca-pc1-for-peer-instructions}
	}
	\QuestionAnswers
	{
		\answer the scores associated to the samples in the dataset
		\answer the number of PCs computed from the dataset
		\correctanswer the loadings associated to $X_1$, $X_2$, and $X_3$
		\answer I do not know
	}
	\QuestionSolution{}
\end{frame}


\begin{frame}{Question E6.10}
	\QuestionKCs{loadings,principal components}
	\QuestionKCsTaxonomies{(1,1)}
	\QuestionBody
	{
		This angle in green depends on? \vspace{0.2cm} \\
		\centering
		\includegraphics[width = 0.45\textwidth]{pca-pc1-for-peer-instructions}
	}
	\QuestionAnswers
	{
		\answer the number of PCs computed from the dataset
		\answer on the relative importance of the various scores
		\correctanswer on the relative importance of the various loadings
		\answer I do not know
	}
	\QuestionSolution{}
\end{frame}


\begin{frame}{Question E6.11}
	\QuestionKCs{principal components analysis}
	\QuestionKCsTaxonomies{(1,1)}
	\QuestionBody{PCA is a useful tool if variables are\ldots}
	\QuestionAnswers
	{
		\correctanswer correlated
		\answer uncorrelated
		\answer independent
		\answer I do not know
	}
	\QuestionSolution{}
\end{frame}


\begin{frame}{Question E6.12}
	\QuestionKCs{principal components analysis}
	\QuestionKCsTaxonomies{(1,1)}
	\QuestionBody{In a PCA analysis, the loading vectors necessarily represent physical factors}
	\QuestionAnswers
	{
		\answer true
		\correctanswer false
		\answer I do not know
	}
	\QuestionSolution{}
\end{frame}


\begin{frame}{Question E6.13}
	\QuestionKCs{principal components analysis}
	\QuestionKCsTaxonomies{(1,1)}
	\QuestionBody{If one needs fewer PCs to get a certain explained variance, then this means that we have \ldots}
	\QuestionAnswers
	{
		\answer a more interpretable model
		\correctanswer a simpler model
		\answer a more complex model
		\answer I do not know
	}
	\QuestionSolution{if a model is not interpretable for any number of PCs then even if we have less PCs we don't increase the interpretability. So the first answer is very often true, but not always. The second, instead, is always true}
\end{frame}


\begin{frame}{Question E6.14}
	\QuestionKCs{scores plots}
	\QuestionKCsTaxonomies{(1,1)}
	\QuestionBody{A scores plot shows \ldots}
	\QuestionAnswers
	{
		\correctanswer the distribution of the samples
		\answer the correlations among the scores
		\answer the correlations among the loadings
		\answer I do not know
	}
	\QuestionSolution{the scores plot shows the distribution of the samples, and thus we can check if there are patterns, groupings, similarities and differences in the samples}
\end{frame}


\begin{frame}{Question E6.15}
	\QuestionKCs{loadings plot}
	\QuestionKCsTaxonomies{(1,1)}
	\QuestionBody{The loadings plot is useful to understand \ldots}
	\QuestionAnswers
	{
		\answer the correlations among the scores
		\answer the correlations among the loadings
		\correctanswer the correlations among the variables
		\answer I do not know
	}
	\QuestionSolution{}
\end{frame}


\begin{frame}{Question E6.16}
	\QuestionKCs{residuals}
	\QuestionKCsTaxonomies{(1,1)}
	\QuestionBody{Residuals plots give information about \ldots}
	\QuestionAnswers
	{
		\answer the distribution of the samples
		\answer the normality of the scores
		\correctanswer the presence of possible outliers and anomalies
		\answer I do not know
	}
	\QuestionSolution{}
\end{frame}


\begin{frame}{Question E6.17}
	\QuestionKCs{bias vs variance}
	\QuestionKCsTaxonomies{(1,1)}
	\QuestionBody
	{
		This plot represents\ldots \vspace{0.2cm} \\
		\centering
		\includegraphics[width = 0.5\textwidth, height = 0.3\textheight]{bias-variance-tradeoff-plot-without-labels}
	}
	\QuestionAnswers
	{
		\correctanswer a bias-variance tradeoff
		\answer the complexity of a model selection problem
		\answer the results of a cross-validation
		\answer I do not know
	}
	\QuestionSolution{}
\end{frame}


\begin{frame}{Question E6.18}
	\QuestionKCs{loadings}
	\QuestionKCsTaxonomies{(1,1)}
	\QuestionBody{Assume to collect daily temperatures measurements from 4 sensors equally spaced in this room. How will the loadings of th 1st PC probably look like?}
	\QuestionAnswers
	{
		\answer $[0.25 \quad 0.25 \quad 0.25 \quad 0.25]$
		\correctanswer $[0.5 \quad 0.5 \quad 0.5 \quad 0.5]$
		\answer $[1 \quad 0 \quad 0 \quad 0]$
		\answer I don't know
	}
	\QuestionSolution{the loadings should be almost the same; moreover we need to have orthonormality, thus we need the Euclidean norm of the vector to be equal to 1}
\end{frame}


\begin{frame}{Question E6.19}
	\QuestionKCs{loadings}
	\QuestionKCsTaxonomies{(1,1)}
	\QuestionBody{Assume to collect daily temperatures measurements from 4 sensors in this room, but assume also that one of the sensors is closer to a window than the rest. How will the loadings of the 1st PC probably look like?}
	\QuestionAnswers
	{
        \answer $[0.5 \quad 0.5 \quad 0.5 \quad 0.5]$
        \correctanswer the loading associated to that sensor will be higher
        \answer the loading associated to that sensor will be smaller
		\answer I do not know
	}
	\QuestionSolution{}
\end{frame}


\begin{frame}{Question E6.20}
	\QuestionKCs{data normalization}
	\QuestionKCsTaxonomies{(1,1)}
	\QuestionBody
	{
		Consider the following dataset. Which components shall we normalize? \vspace{0.1cm} \\
		\centering
		\includegraphics[width = 0.6\textwidth]{scatterplots-of-dataset-with-5-variables}
		\vspace{-2cm}
	}
	\QuestionAnswers
	{
		\answer none
		\correctanswer all
		\answer only the 2nd and 3rd
		\answer I do not know
	}
	\QuestionSolution{}
\end{frame}


\begin{frame}{Question E6.21}
	\QuestionKCs{SVD}
	\QuestionKCsTaxonomies{(1,1)}
	\QuestionBody{In a SVD, are both the $U$ and $V$ matrices operating from / to the same spaces?}
	\QuestionAnswers
	{
		\answer yes, always
		\answer no, never
		\correctanswer sometimes
		\answer I do not know
	}
	\QuestionSolution{}
\end{frame}


\begin{frame}{Question E6.22}
	\QuestionKCs{PCA}
	\QuestionKCsTaxonomies{(1,1)}
	\QuestionBody{When doing a PCA, is the order of the data rows important?}
	\QuestionAnswers
	{
		\answer yes, always
		\correctanswer no, never
		\answer sometimes
		\answer I do not know
	}
	\QuestionSolution{}
\end{frame}


\begin{frame}{Question E6.23}
	\QuestionKCs{principal component analysis}
	\QuestionKCsTaxonomies{(1,1)}
	\QuestionBody
	{
	    On which of these datasets will PCA work \emph{worse}? \vspace{0.2cm} \\
	    \centering
	    \includegraphics[width = 0.5\textwidth]{circular-dataset-and-cross-dataset}
	    \vspace{-1cm}
	}
	\QuestionAnswers
	{
		\answer in A
		\answer in B
		\correctanswer in both
		\answer I do not know
	}
	\QuestionSolution{}
\end{frame}


\begin{frame}{Question E6.24}
	\QuestionKCs{PCA}
	\QuestionKCsTaxonomies{(1,1)}
	\QuestionBody
	{
		How would you score the pilots below, based on their abilities to take off AND land? \vspace{0.1cm} \\
		\centering
		\includegraphics[width = 0.4\textwidth]{taking-off-score-vs-landing-score}
% 		\newcommand{\InsertImageAt}[5] % path / height / width / xshift / yshift
% 		\InsertImageAt{taking-off-score-vs-landing-score}{3}{5}{5}{-3}
	}
	\QuestionAnswers
	{
		\answer select one specific component
		\answer use the average of the two components
		\correctanswer use a non-symmetric combination of the two components
		\answer I do not know
	}
	\QuestionSolution{}
\end{frame}


