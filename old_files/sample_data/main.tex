% example of a multiple-choice question with minimalistic information provided
\begin{IndexedQuestion}
	\QuestionContentUnits{time constant, first order systems}
	\QuestionTaxonomyLevels{u2, e1}
	\QuestionType{multiple choice}
	\QuestionBody{Test: Consider the system $\dot{y} = -0.5 y + 2.3 u$, $y_{0} = 1$, and $\left| u(t) \right| \leq 2$ for every $t$. Assume moreover that from $t = 10$, $u(t) = 0$. Then we are sure that $y(t) \approx 0$ at the earliest starting from \ldots}
	\QuestionPotentialAnswers{\answer $t = 10$\correctanswer $t = 20$\answer $t = 30$\answer $t = 40$\answer none of the above\answer I do not know}
\end{IndexedQuestion}


% example of a question where one has compiled all the compilable fields
\begin{IndexedQuestion}
	\QuestionAuthorEmail{myemail@mydomain.edu}
	\QuestionContentUnits{sums, multiplications} % must be a comma separated list
	\QuestionTaxonomyLevels{e1, u1} % must be a comma separated list
	\QuestionType{multiple choice} % [ multiple choice | open | numeric ]
	\QuestionBody{$( 2 + 2 ) \cdot 1$ equals to?}
% 	\QuestionImage{../myfilepath/myimage.jpg} % keep commented if you don't have an image
	\QuestionPotentialAnswers{ \answer 2 \correctanswer 4 \answer I don't know } % this field is meaningful only for ``multiple choice'' questions
	\QuestionSolutions{you can arrive at the solution thinking at $1 + 1 = 2$ twice}
	\QuestionNotesForTheTeachers{if they are less than 1 year old then probably they won't answer}
	\QuestionNotesForTheStudents{remember that $1 + 1 = 2$}
	\QuestionFeedbackForTheStudents{if you answered in this way, blabla, if not, blabla}
	\QuestionDisclosability{everybody} % [ only me, only teachers, everybody ]
	\QuestionSolutionDisclosability{only teachers} % [ only me, only teachers, everybody ]
\end{IndexedQuestion}


% example of a complete contents map -- always keep the ``ifshowcontentsmap .. \fi''
\ifshowcontentsmap
\begin{ContentsMap}
	%
	\begin{DevelopedContents}
		content unit A & uA, eA \\
		content unit B & uB, eB \\
		\vdots & \vdots \\
	\end{DevelopedContents}
	%
	\begin{PrerequisiteContents}
		content unit C & uC, eC \\
		content unit D & uD, eD \\
		\vdots & \vdots \\
	\end{PrerequisiteContents}
	%
	\begin{ContentsRelationships}
		%
		\begin{WhatIsNecessaryForWhatRelationships}
			CU X & uX, eX & CU Y & uY, eY \\
			\vdots & \vdots & \vdots & \vdots \\
		\end{WhatIsNecessaryForWhatRelationships}
		%
		\begin{WhatIsUsefulForWhatRelationships}
			CU X & uX, eX & CU Y & uY, eY \\
			\vdots & \vdots & \vdots & \vdots \\
		\end{WhatIsUsefulForWhatRelationships}
		%
		\begin{WhatGeneralizesWhatRelationships}
			CU X & CU Y \\
			\vdots & \vdots \\
		\end{WhatGeneralizesWhatRelationships}
		%
		\begin{WhatIsASynonymOfWhatRelationships}
			CU X & CU Y \\
			\vdots & \vdots \\
		\end{WhatIsASynonymOfWhatRelationships}
		%
		\begin{WhatIsDirectlyLogicallyConnectedToWhatRelationships}
			CU X & uX, eX & CU Y & uY, eY \\
			\vdots & \vdots & \vdots & \vdots \\
		\end{WhatIsDirectlyLogicallyConnectedToWhatRelationships}
		%
	\end{ContentsRelationships}
	%
\end{ContentsMap}
\fi


\end{document}
% -------------------------------------------------------------- %

