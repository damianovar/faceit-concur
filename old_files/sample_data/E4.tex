\begin{frame}{Question E4.1}
	\QuestionKCs{overfitting,underfitting}
	\QuestionKCsTaxonomies{(1,1)}
	\QuestionNotes{}
	\QuestionBody{If a model shows good MSE in training and bad MSE in testing, then it is \ldots}
	\QuestionAnswers
	{
		\correctanswer overfitting
		\answer underfitting
		\answer correctly fitting
		\answer I don't know
	}
\end{frame}


\begin{frame}{Question E4.2}
	\QuestionKCs{overfitting,underfitting}
	\QuestionKCsTaxonomies{(1,1),(1,1)}
	\QuestionNotes{}
	\QuestionBody{If a model shows bad MSE in training and bad MSE in testing, then it is \ldots}
	\QuestionAnswers
	{
		\answer overfitting
		\correctanswer underfitting
		\answer correctly fitting
		\answer I don't know
	}
\end{frame}


\begin{frame}{Question E4.3}
	\QuestionKCs{overfitting,underfitting}
	\QuestionKCsTaxonomies{(1,1),(1,1)}
	\QuestionNotes{}
	\QuestionBody{If a model shows bad MSE in training and good MSE in testing, then it is \ldots}
	\QuestionAnswers
	{
		\answer overfitting
		\answer underfitting
		\correctanswer I think this happens only by chance
		\answer I don't know
	}
\end{frame}


\begin{frame}{Question E4.4}
	\QuestionKCs{cross validation}
	\QuestionKCsTaxonomies{(1,1)}
	\QuestionNotes{}
	\QuestionBody{What is the drawback of doing a LOO-CV instead of a 10-fold-CV? We will \ldots}
	\QuestionAnswers
	{
		\answer compute less accurate statistics
		\correctanswer use more computational power
		\answer have numerical problems
		\answer I don't know
	}
\end{frame}


\begin{frame}{Question E4.5}
	\QuestionKCs{model order selection}
	\QuestionKCsTaxonomies{(1,1)}
	\QuestionNotes{}
	\QuestionBody{A model should be more flexible than parsimonious}
	\QuestionAnswers
	{
		\answer true
		\answer false
		\correctanswer depends
		\answer I don't know
	}
\end{frame}


\begin{frame}{Question E4.6}
	\QuestionKCs{data analysis workflow}
	\QuestionKCsTaxonomies{(1,1)}
	\QuestionNotes{}
	\QuestionBody{Solving a data science problem should start with \ldots}
	\QuestionAnswers
	{
		\correctanswer listing the a-priori information
		\answer designing the experiments
		\answer selecting the model order
		\answer selecting the model structure
		\answer I don't know
	}
\end{frame}


\begin{frame}{Question E4.7}
	\QuestionKCs{model order selection}
	\QuestionKCsTaxonomies{(1,1)}
	\QuestionNotes{}
	\QuestionBody{When constructing model structures based on physics laws, we should be as accurate and detailed as possible and thus include all the effects we may think may affect the data}
	\QuestionAnswers
	{
		\answer true
		\correctanswer false
		\answer I don't know
	}
\end{frame}


\begin{frame}{Question E4.8}
	\QuestionKCs{linear systems}
	\QuestionKCsTaxonomies{(1,1)}
	\QuestionNotes{}
	\QuestionBody{If the responses to a generic square wave are symmetric then the system is \ldots}
	\QuestionAnswers
	{
		\correctanswer linear
		\answer nonlinear
		\answer I don't know
	}
\end{frame}


\begin{frame}{Question E4.9}
	\QuestionKCs{model order selection,maximum likelihood}
	\QuestionKCsTaxonomies{(1,1),(1,1)}
	\QuestionNotes{}
	\QuestionBody{To estimate which model order shall be used, the most reliable method is using ML}
	\QuestionAnswers
	{
		\answer true
		\correctanswer false
		\answer depends
		\answer I don't know
	}
\end{frame}


\begin{frame}{Question E4.10}
	\QuestionKCs{MSE}
	\QuestionKCsTaxonomies{(1,1)}
	\QuestionNotes{}
	\QuestionBody{The MSE of an estimator depends on \ldots}
	\QuestionAnswers
	{
		\answer the density of the measurements
		\answer the estimand
		\correctanswer both
		\answer I don't know
	}
\end{frame}


\begin{frame}{Question E4.11}
	\QuestionKCs{MSE}
	\QuestionKCsTaxonomies{(1,1)}
	\QuestionNotes{}
	\QuestionBody{The MSE is defined over the measure of \ldots}
	\QuestionAnswers
	{
		\correctanswer the data
		\answer the estimand
		\answer both
		\answer I don't know
	}
\end{frame}


\begin{frame}{Question E4.12}
	\QuestionKCs{MSE}
	\QuestionKCsTaxonomies{(1,1)}
	\QuestionNotes{}
	\QuestionBody{The MSE can be divided into the sum of}
	\QuestionAnswers
	{
		\answer bias + variance
		\correctanswer bias$^2$ + variance
		\answer bias + variance$^2$
		\answer I don't know
	}
\end{frame}


\begin{frame}{Question E4.13}
	\QuestionKCs{underfitting}
	\QuestionKCsTaxonomies{(1,1)}
	\QuestionNotes{}
	\QuestionBody{Underfitting is associated to \ldots}
	\QuestionAnswers
	{
		\correctanswer too much bias
		\answer too much variance
		\answer I don't know
	}
\end{frame}


\begin{frame}{Question E4.14}
	\QuestionKCs{overfitting}
	\QuestionKCsTaxonomies{(1,1)}
	\QuestionNotes{}
	\QuestionBody{Overfitting is associated to \ldots}
	\QuestionAnswers
	{
		\answer too much bias
		\correctanswer too much variance
		\answer I don't know
	}
\end{frame}


\begin{frame}{Question E4.15}
	\QuestionKCs{variance}
	\QuestionKCsTaxonomies{(1,1)}
	\QuestionNotes{}
	\QuestionBody{The variance of an estimator is connected to \ldots}
	\QuestionAnswers
	{
		\answer its sensitivity to changes in the parameters
		\correctanswer its sensitivity to changes in the dataset
		\answer its sensitivity to changes in the model structure
		\answer I don't know
	}
\end{frame}


\begin{frame}{Question E4.16}
	\QuestionKCs{overfitting}
	\QuestionKCsTaxonomies{(1,1)}
	\QuestionNotes{}
	\QuestionBody{we avoid overfitting even if results are better in the training set because we want \ldots}
	\QuestionAnswers
	{
		\answer models that are structurally simple
		\answer the variance of the estimator to be small
		\correctanswer models that are good in the test set
		\answer I don't know
	}
\end{frame}


\begin{frame}{Question E4.17}
	\QuestionKCs{variance}
	\QuestionKCsTaxonomies{(1,1)}
	\QuestionNotes{}
	\QuestionBody{more measurement noise implies that the variance of the estimator will be \ldots}
	\QuestionAnswers
	{
		\correctanswer higher
		\answer the same
		\answer lower
		\answer I don't know
	}
\end{frame}



